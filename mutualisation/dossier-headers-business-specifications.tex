
% dossier-headers-business-contract.tex
% https://www.uweziegenhagen.de/?p=2224
% https://www2.icp.uni-stuttgart.de/~icp/mediawiki/images/5/51/Latex-template.tex


\RequirePackage[l2tabu,orthodox]{nag} % turn on warnings because of bad style
%\documentclass[a4paper,11pt]{scrartcl}
\documentclass[a4paper,11pt]{article}
%\usepackage{common/mutualisation/default-packages}
\usepackage{common/business/business-invoice}


\usepackage[french]{babel}
%\usepackage[a5paper,top=2cm, bottom=2cm, left = 2.5cm, right=2cm]{geometry}

\usepackage{graphicx}
\usepackage{fancyhdr}
%\usepackage[headheight=45pt]{geometry}

\usepackage{geometry}
%\geometry{hmargin=1.0cm,vmargin=2.5cm}
\geometry{headheight=45pt, hmargin=1.0cm,vmargin=3.0cm}
%\geometry{headheight=55pt}


\usepackage{longtable}


\usepackage[utf8]{inputenc}

\usepackage[T1]{fontenc}        % Tries to use Postscript Type 1 Fonts for better rendering
\usepackage{lmodern}            % Provides the Latin Modern Font which offers more glyphs than the default Computer Modern
\usepackage[intlimits]{amsmath} % Provides all mathematical commands

\usepackage{hyperref}           % Provides clickable links in the PDF-document for \ref
\usepackage{grffile}            % Allow you to include images (like graphicx). Usage: \includegraphics{path/to/file}

% Allows to set units
\usepackage[ugly]{units}        % Allows you to type units with correct spacing and font style. Usage: $\unit[100]{m}$ or $\unitfrac[100]{m}{s}$

% Additional packages
\usepackage{url}                % Lets you typeset urls. Usage: \url{http://...}
\usepackage{breakurl}           % Enables linebreaks for urls
\usepackage{xspace}             % Use \xpsace in macros to automatically insert space based on context. Usage: \newcommand{\es}{ESPResSo\xspace}
\usepackage{xcolor}             % Obviously colors. Usage: \color{red} Red text
\usepackage{booktabs}           % Nice rules for tables. Usage \begin{tabular}\toprule ... \midrule ... \bottomrule


\usepackage{multicol}
\setlength{\columnsep}{0.5cm}

\usepackage{tikz}
\usepackage{tikzpagenodes}


\usepackage{enumitem}
\setlist[itemize]{align=parleft,left=0pt..1em}
\setlist[enumerate]{align=parleft,left=0pt..1em}


\usepackage[framemethod=tikz]{mdframed}
\usepackage{hhline}


%%%%%%%%%%%%%%%%%%%%%%%%%%%%%%%%%%%%%%%%%%%%%%%%%%%%%
\renewcommand{\arraystretch}{1.2}



\usetikzlibrary{calc}
\usetikzlibrary{positioning}


\usetikzlibrary{positioning,arrows.meta}
\usetikzlibrary{shapes}







\usepackage{common/business/business-utils}




% --------------- pointillés
\usepackage{tocloft}
\renewcommand{\cftsecleader}{\cftdotfill{\cftdotsep}}

%%%%%%%%%%%%%%%%%%%% headers


\def\specpagebgcolor{general@best@navy}
\newcommand{\mainpagestyle}[2]{
  \renewcommand{\headrulewidth}{0pt}
  \fancypagestyle{plain}{%
  \fancyhead[R]{
      \begin{tikzpicture}[remember picture,overlay]
          \draw [draw=\specpagebgcolor!60,fill=\specpagebgcolor!60,opacity=0.65] (0,0) -- (0,1.75) --++(-2,0) -- cycle;
          \draw [draw=\specpagebgcolor!60,fill=\specpagebgcolor!60,opacity=0.65] (0,0.7) -- (0,1.75) --++(-2.5,0) -- cycle;
          \draw [draw=\specpagebgcolor!100,fill=\specpagebgcolor!100,opacity=0.65] (0,0.1) -- (0,1.75) --++(-1.2,0) -- cycle;
          \draw[text=white,xshift=-0.375cm,yshift=1.35cm] node {\bfseries \thepage};
      \end{tikzpicture}
  }
    \fancyhead[L]{
      \logoMDB
      %\vspace{.05cm}
    }%
    \fancyfoot[C]{}
    \fancyfoot[R]{
      \textcolor{general@best@blue}{#1} 
      \textcolor{general@best@gray}{| #2}
    }%
  }
  \pagestyle{plain}
}




%%%%%%%%%%%%%%%%%%%%%%%%%%%%%%%%%%%%%%%%%%%




\def\specheadercolora{general@business@blueAfandi}
\def\specheadercolorb{general@business@blueAfandi}

\newcommand{\basepagea}[5]{
  \begin{titlepage}
    \begin{tikzpicture}
        \draw[fill=\specheadercolora, draw=\specheadercolora] (0,-1.2) rectangle (.87\textwidth,1.2);  
        \node[fill=\specheadercolora,text=\specheadercolora, draw=\specheadercolora,circle,minimum size=3.25cm,inner sep=0pt] at (0,0) {\fontsize{50}{60}\selectfont MD};
        \node[fill=white,text=\specheadercolora, draw=\specheadercolora,circle,minimum size=3.1cm,inner sep=0pt] at (0,0) {\fontsize{40}{60}\selectfont MD};
        %\node[text=general@business@blueCiel,draw,circle,minimum size=1cm,inner sep=0pt] at (2,0) {\fontsize{50}{60}\selectfont MD};
        \node(boundbox) [fill=white,scale=1.3,draw=gray!25,xshift=10.76cm] (logoscope){
            \logoMDC
          };
    \end{tikzpicture}
  
    \vspace{3cm}
  
    \begin{minipage}{0.50\textwidth}   %left column
        \vspace{2cm}
        \includegraphics[width=.8\linewidth]{#1}
        
        \begin{flushleft}
            \textcolor{\specheadercolorb}{\Large \bfseries #2}
  
            \vspace{.5cm}
  
            \textcolor{\specheadercolorb}{\Large \bfseries #3}
        \end{flushleft}
        \vspace{1cm}
        \begin{flushright}
            \textit{#4}
        \end{flushright}
        \vspace{2cm}
    \end{minipage}
    \textcolor{general@best@maroon}{\hfill\vline width 2pt\hfill}
    \begin{minipage}{0.40\textwidth} %right column
        \Large \textbf{Objectifs}
  
        \vspace{1cm}
        \normalsize
        #5
    \end{minipage}
  
  \end{titlepage}
  \tableofcontents
}



\newcommand{\textgr}[1]{\textcolor{gray}{#1}}



%------------------------------------------------------------------------------


% ----------------------------------------------------------------------
\def\specdelayacolor{general@best@blue}
\def\specdelaybcolor{general@best@silver}
\newcommand{\specdelaya}[1]&
      \textcolor{white}{\bfseries échéance}\\\hline
      #1
    \end{longtable}
}
\newcommand{\spdr}[5]{%
\multicolumn{1}{|C{.05\textwidth}}{\footnotesize#1} &
\multicolumn{1}{|L{.52\textwidth}}{\footnotesize#2} &
\multicolumn{1}{|C{.150\textwidth}}{\bfseries\footnotesize#3}&
\multicolumn{1}{|C{.055\textwidth}}{\bfseries\footnotesize\cellcolor{\specdelayacolor}\textcolor{white}{#4}}&
\multicolumn{1}{|C{.100\textwidth}|}{\textcolor{red}{\bfseries\footnotesize#5}}\\\hline
}
\newcommand{\spdtr}[1]{%
\rowcolor{\specdelaybcolor}\multicolumn{5}{l}{\textsc{#1}}\\\hline
}
\def\delayurgent{\bwarning{urgent}}
\def\delayinter{-}%{\binfo{intermédiaire}}
\def\delayturgent{\bdanger{très urgent}}
% ----------------------------------------------------------------------

\def\exbesoinacolor{general@best@blue}
\def\exbesoinbcolor{general@best@silver}
\newcommand{\exbesoina}[1]{%
    \begin{longtable}{ccc}
      \arrayrulecolor{\exbesoinbcolor}
      \hline 
      \rowcolor{white}\multicolumn{3}{|c|}{ \textcolor{\exbesoinacolor}{\textbf{expression du besoin}} }  \\
      %\rowcolor{\exbesoinacolor} \bfseries \textcolor{white}{$\boldsymbol{\#}$}&
      \rowcolor{\exbesoinacolor} \bfseries
      \textcolor{white}{\bfseries besoin}&
      \textcolor{white}{\bfseries priorité}&
      %\textcolor{white}{\bfseries \%}&
      \textcolor{white}{\bfseries coût horaire}\\\hline
      #1
    \end{longtable}
}
\newcommand{\expbr}[3]{%
%\multicolumn{1}{|C{.05\textwidth}}{#1} &
\multicolumn{1}{|L{.64\textwidth}}{\footnotesize#1} &
\multicolumn{1}{|C{.145\textwidth}}{\bfseries\footnotesize#2}&
%\multicolumn{1}{|C{.045\textwidth}}{\bfseries #3}&
\multicolumn{1}{|C{.150\textwidth}|}{\textcolor{red}{\bfseries\footnotesize#3}}\\\hline
}
\newcommand{\expbtr}[1]{%
\rowcolor{\exbesoinbcolor}\multicolumn{3}{l}{\textsc{#1}}\\\hline
}

% ----------------------------------------------------------------------




% ---------------------------------------------------------------------- block

\tikzset{
  %node distance=5mm and 0mm,
  boite/.style={
    fill=general@best@navy,draw,align=left,
    font=\footnotesize\bfseries,text=white,
  },
  boite coins ronds/.style={boite,rounded corners=3pt},
  boite circulaire/.style={boite,circle},
  fleche/.style={
    line cap=round,-latex,line width=0.25mm,
    draw=blue!50!red!30,
  },
  eqfleche/.style={
    line cap=round,latex-latex,line width=0.25mm,
    draw=blue!50!orange!30,
  },
  grosse fleche/.style={fleche,line width=1mm},
  grosse eqfleche/.style={eqfleche,line width=1mm},
}


%%%%%%%%%%%%%%%%%%%%%%%%%%%%%%%%%%%%%%%%%%%%%%%%%%%%%% Style des itemize

\newcommand{\itemclass}[2] % #1 = couleur ; #2 = fonte
{

  
  %\setlist[itemize,1]{label={\color{#1}\raisebox{.25\height}{$\bullet$}},itemsep=0pt,topsep=0pt}
	%\setlist[itemize,1]{label={\color{#1}$\circ$}}
	%\setlist[itemize,2]{label={\color{#1}$\rightarrow$}}

  %\setlist[itemize,2]{label={\color{#1}a}}
  \setlength{\fboxsep}{2pt}
	\setlist[enumerate,1]
	{%
		label=\fcolorbox{#1}
		{#1!100}
		{\color{white}#2\bfseries\normalsize\arabic*}
	}
	\setlist[enumerate,2]{label=\textcolor{#1}{#2\normalsize\alph*.}}
}
\itemclass{general@best@navy}{}

