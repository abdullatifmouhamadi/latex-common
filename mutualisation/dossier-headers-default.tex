
%\documentclass{report}
\documentclass[ams,openany,10pt,presentation,utf8]{common/themes/dossiers/dossiers}


\usepackage[object=vectorian]{pgfornament} %%  http://altermundus.com/pages/tkz/ornament/index.html



%%%%%%%%%%%%%%%%%%%%%%%%%%%%%%%%%%%%%%%%%%%%%%%%%%%%%%%%%%%%%%%% MARGES 
% method 2
\usepackage{geometry}
\geometry{hmargin=1.0cm,vmargin=1.5cm}


%%%%%%%%%%%%%%%%%%%%%%%%%%%%%%%%%%%%%%%%%%%%%%%%%%%%%%%%%%%%%%%%


 

%%%%%%%%%%%%%%%%%%%%%%%%%%%%%%%%%%%%%%%%%%%%%%%%%%%%%%%%%%%%%%%% TIMING
\usepackage{tikz-timing}
\usetikzlibrary{positioning,arrows.meta}
\usetikztiminglibrary[new={char=Q,reset char=R}]{counters}
\usetikztiminglibrary{nicetabs} 

%%%%%%%%%%%%%%%%%%%%%%%%%%%%%%%%%%%%%%%%%%%%%%%%%%%%%%%%%%%%%%%%

%%%%%%%%%%%%%%%%%%%%%%%%%%%%%%%%%%%%%%%%%%%%%%%%%%%%%%%%%%%%%%%% HEADERS/FOOTERS
\usepackage{fancyhdr}
\pagestyle{fancy}



\fancyhead[LE,RO]{\textsl{}}
\fancyhead[LO,RE]{\textsl{}}

%\fancyhead[LE,RO]{\textsl{\rightmark}}
%\fancyhead[LO,RE]{\textsl{\leftmark}}
%\fancyhead[LE,RO]{\textsl{}}
\fancyfoot[C]{\thepage}


\fancyfoot[L]{MAOREDEV}
%\fancyfoot[C]{Electronique \& Communication}
%\fancyfoot[R]{\thepage}
\fancyfoot[R]{\today }
%\renewcommand{\headrulewidth}{0.4pt}
\renewcommand{\headrulewidth}{0.0pt}
\renewcommand{\footrulewidth}{0.4pt}

%%%%%%%%%%%%%%%%%%%%%%%%%%%%%%%%%%%%%%%%%%%%%%%%%%%%%%%%%%%%%%%%




%%%%%%%%%%%%%%%%%%%%%%%%%%%%%%%%%%%%%%%%%%%%%%%%%%%%%%%%%%%%%%%%

%%%%%%%%%%%%%%%%%%%%%%%%%%%%%%%%%%%%%%%%%%%%%%%%%%%%%%%%%%%%%%%%



\usepackage{amsfonts,amsmath,amssymb,amsthm}
\usepackage{aligned-overset}

\usepackage{mathptmx}
\usepackage{relsize}

\usepackage[T1]{fontenc}
\usepackage{graphicx}
\usepackage{ifthen}

\usepackage[Lenny]{fncychap}


\usepackage{listings,multicol}
\usepackage{color}

% print subsubsection
\setcounter{tocdepth}{5}
\setcounter{secnumdepth}{5}
% fin print subsubsection


\usepackage{minitoc}
\setcounter{minitocdepth}{3}
%\setcounter{minitocdepth}{2}
\setlength{\mtcindent}{24pt}
\renewcommand{\mtcfont}{\small\rm}
\renewcommand{\mtcSfont}{\small\bf}

\usepackage{hyperref}
\hypersetup{pdftex,colorlinks=true,allcolors=blue}
\usepackage{hypcap}


\usepackage{type1cm}
\usepackage{lettrine}
\usepackage{scrextend}
\usepackage{blindtext}
%\usepackage[svgnames, x11names]{xcolor}



\usepackage[utf8]{inputenc}
\usepackage{lmodern,textcomp}

\usepackage[french]{babel}
\mtcselectlanguage{french}

\usepackage{multido}
\RequirePackage{framed}
\usepackage{fancyhdr}
\usepackage{fancybox}
\usepackage{fancybox,xcolor}
\usepackage{blindtext}
\usepackage{xcolor, colortbl} %kleurtjes voor table

\usepackage{hhline}
\usepackage{makecell} % https://tex.stackexchange.com/questions/2441/how-to-add-a-forced-line-break-inside-a-table-cell



\usepackage{answers}

%\usepackage{mdframed}

\usepackage[framemethod=TikZ]{mdframed}

\usepackage{pstricks,pst-grad}

% ntheorem
\makeatletter
\let\th@plain\relax
\makeatother
%\usepackage[thref]{ntheorem}
%\usepackage[amsmath,thref]{ntheorem}
\usepackage{empheq}
\usepackage{thmtools}
\usepackage{subfig}
\usepackage{hyperref}%
\usepackage{systeme,mathtools}



%%%%%%%%%%%%%%%%%%%%%%%%%%%%%%%%%%%%%%%%%%%%%%%%%%%%%%%%%%%%%%%% ARRAY 
\usepackage{array,multirow,makecell}
\renewcommand{\arraystretch}{1.2}
%\setcellgapes{1pt}
%\makegapedcells
%\newcolumntype{R}[1]{>{\raggedleft\arraybackslash }b{#1}}
%\newcolumntype{L}[1]{>{\raggedright\arraybackslash }b{#1}}
%\newcolumntype{C}[1]{>{\centering\arraybackslash }b{#1}}
%%%%%%%%%%%%%%%%%%%%%%%%%%%%%%%%%%%%%%%%%%%%%%%%%%%%%%%%%%%%%%%%

%%%%%%%%%%%%%%%%%%%%%%%%%%%%%%%%%%%%%%%%%%%%%%%%%%%%%%%%%%%%%%%% PERSONNAL
\usepackage{common/lesson/headers}
\usepackage{common/lesson/blocks}
\usepackage{common/lesson/theorems}
%\usepackage{common/lesson/tests}
\usepackage{common/lesson/ccf}
\usepackage{common/lesson/pretty}

\usepackage{common/images/nodes}
\usepackage{common/lesson/tables}

\definecolor{Gray}{gray}{0.9}

%%%%%%%%%%%%%%%%%%%%%%%%%%%%%%%%%%%%%%%%%%%%%%%%%%%%%%%%%%%%%%%%

% moi-même
\renewcommand{\thesection}{\arabic{section}} % Remove chapter number to sections

\usepackage{chngcntr}
\usepackage{xhfill}


% Circle
\usepackage{pict2e,picture}
\newsavebox\CBox
\newlength\CLength
\def\competence#1{\sbox\CBox{#1}%
  \ifdim\wd\CBox>\ht\CBox \CLength=\wd\CBox\else\CLength=\ht\CBox\fi
    \makebox[1.1\CLength]{\makebox(0,1.1\CLength){\put(0,0){\circle{1.0\CLength}}\put(0,0){\circle{0.9\CLength}}}%
    \makebox(0,1.1\CLength){\put(-.3\wd\CBox,0){\tiny\textbf{#1}}}}}
% End circle


%----- Ensembles : entiers, reels, complexes -----
\newcommand{\Nn}{\mathbb{N}} \newcommand{\N}{\mathbb{N}}
\newcommand{\Zz}{\mathbb{Z}} \newcommand{\Z}{\mathbb{Z}}
\newcommand{\Qq}{\mathbb{Q}} \newcommand{\Q}{\mathbb{Q}}
\newcommand{\Rr}{\mathbb{R}} \newcommand{\R}{\mathbb{R}}
%\newcommand{\Cc}{\mathbb{C}} \newcommand{\C}{\mathbb{C}}
\newcommand{\Kk}{\mathbb{K}} \newcommand{\K}{\mathbb{K}}


\newcommand{\sk}{\vspace{0.2 cm}}
\newcommand{\A}[1]{{$\backslash${\tt #1}}}
\newcommand{\nsp}{\mbox{\hspace{-1 cm}}}

\title{\LARGE \textbf{BTS Systèmes Numériques} \\ \normalsize (Option \'Electronique \& Communication) \\ \Large CCF 2020-2022 \\ Sujet X}
%\author{ABDULLATIF Mouhamadi}
\author{CFAI Poitou-Charentes}
\date{\today}



%\def\PRINTMODE{PROFESSEUR}
\def\PRINTMODE{ELEVE}


\newcommand{\TEACHERBLOC}{\ifthenelse{\equal{\PRINTMODE}{PROFESSEUR}}}



%%%%%%%%%%%%%%%%%%%%%%%%%%%%%%%%%%%%%%%%%%%%%%%%%%%%%%%%% remove chapter
\usepackage{titlesec}
%\titleformat{\chapter}[display]{\normalfont\bfseries}{}{0pt}{\Huge}
  %{\normalfont\bfseries}{}{0pt}{\Huge}
\usepackage{ulem}
\usepackage[subfigure]{tocloft}

\usepackage[final]{pdfpages}
\usepackage{pdflscape}






\AtBeginDocument{
    \renewcommand{\labelitemi}{\textbullet}
    \renewcommand{\labelitemi}{$\circ$}
    \renewcommand{\labelitemii}{$\circ$}
    \renewcommand{\labelitemiii}{$\circ$}
    \renewcommand{\labelitemiv}{$\circ$}
}



%%%%%%%%%%%%%%%%%%%%%%%%%%%%%%%%%%%%%%%%%%%%%%%%%%%%%%%%%%%%%%%% IMAGE FRAME 
\usepackage{efbox,graphicx}
%_______________________________________________________________

\usepackage{csvsimple}

\usepackage{longtable}


\newcommand{\redt}[1]{\textcolor{general@best@maroon}{\textbf{#1}}}

%yay -S texlive-tikz-uml
\usepackage{tikz-uml}
%\usepackage{umlpackage}
\usepackage{uml}






\newcommand{\horizb}{%
  \nointerlineskip \vspace{.5\baselineskip}\hspace{\fill}
  {\color{general@best@orange}
    \resizebox{0.75\linewidth}{2ex}
    {{%
    {\begin{tikzpicture}
    \node  (C) at (0,0) {};
    \node (D) at (9,0) {};
    \path (C) to [ornament=89] (D);
    \end{tikzpicture}}}}}%
    \hspace{\fill}
    \par\nointerlineskip \vspace{.5\baselineskip}
  }



%----------------------------------------------------------------------
%.25-.7

\def\sourcelinkacolor{general@best@aqua}
\newcommand{\sourcelinka}[1]{%
  \begin{tabular}{cccc}
    \arrayrulecolor{\sourcelinkacolor}
    \hline
    %\hline 
     %\rowcolor{\sourcelinkacolor}\multicolumn{1}{C{.25\textwidth}}{ \textbf{titre} } & \multicolumn{1}{C{.7\textwidth}|}{ \textbf{lien} } \\
    #1
    %\hline
  \end{tabular}
}
\newcommand{\slar}[2]{%
%#1&#2&#3\\\hline
\multicolumn{1}{L{.25\textwidth}}{ \cellcolor{\sourcelinkacolor}#1 } & \multicolumn{1}{L{.7\textwidth}|}{ \footnotesize\url{#2} } \\\hline

%\cellcolor{general@best@aqua}{#1}&\footnotesize\url{#2}\\%\hline
}



% ----------------------------------------------------------------------

\newcommand{\bubblea}[1]{
  \begin{tikzpicture}
    \node[fill=general@best@teal,rounded corners]
    {\textcolor{white}{\bfseries#1}};
    \end{tikzpicture}
}