\usepackage{common/mutualisation/default-packages}




\usetikzlibrary{decorations.pathmorphing, patterns,shapes,arrows}
\usetikzlibrary{calc}
\usetikzlibrary{positioning}
\usetikzlibrary {patterns.meta}
\usepackage[framemethod=tikz]{mdframed}
\usepackage{hhline}


%\usepackage[T1]{fontenc}
\usepackage{libertine}
\renewcommand{\familydefault}{\sfdefault}
\usepackage{microtype}
\pagenumbering{gobble}
\CutLine{3}
%\usepackage{xcolor}
\renewcommand{\sectfont}{\large\sffamily\bfseries\color{blue}}


\usepackage{xcolor}
\usepackage{dashrule}


\usepackage{environ}
\usepackage{eso-pic}



\usepackage{hyperref}
\hypersetup{
    colorlinks=true,
    linkcolor=blue,
    filecolor=magenta,      
    urlcolor=gray!32,
}


%%%%%%%%%%%%%%%%%%%%%%%%%%%%%%%%%%%%%%%%%%%%%%%%%%%%%%%%%%%%%%%%%%%%%%%%%

%\usepackage{titlesec}

%\titleformat*{\section}{\Large\color{red}}



\usepackage{setspace}

\usepackage{geometry}
%\geometry{hmargin=0.25cm,vmargin=0.25cm}
\geometry{hmargin=0.0cm,vmargin=0.0cm}



%%%%%%%%%%%%%%%%%%%%%%%%%%%%%%%%%%%%%%%%%%%%%%%%%%%%%%%%%%%%%%%%%%%%%%%%%

%https://tex.stackexchange.com/questions/104072/tikz-custom-zigzag-pattern
%https://tex.stackexchange.com/questions/42003/decorating-with-random-steps-a-filling-pattern




\pgfdeclarepatternformonly{zigzagpat}
{\pgfpointorigin}
{\pgfpoint{8cm}{2cm}} % Upper right corner so the box is 8x2 mm
{\pgfpoint{4cm}{0.75cm}} % The tile size is smaller than the picture itself 
                      % So the teeth are getting closer
{
\pgfdecoration{{zigzag}{\pgfdecoratedpathlength}{%Setting up the decoration for the full path
                        \pgfdecorationsegmentlength=2cm      % The zigzag thread width
                        \pgfdecorationsegmentamplitude=0.5cm   % The zigzag thread height
                        \pgfsetlinewidth{8pt}
                        %\pgfsetcolor{gray}
                   }
              }
\pgfpathmoveto{\pgfpoint{0cm}{1cm}}        % So first we move 1 mm up
\pgfpathlineto{\pgfpoint{8cm}{1cm}}        % And draw a straight horizontal line to the end of our box
\endpgfdecoration                          % Close the decoration environment
\pgfusepathqstroke                         % Now draw the decorated path
}

%%

\pgfdeclarepatternformonly{stars}
{\pgfpointorigin}{\pgfpoint{1cm}{1cm}}
{\pgfpoint{1cm}{1cm}}
{
\pgftransformshift{\pgfpoint{.5cm}{.5cm}}
\pgfpathmoveto{\pgfpointpolar{0}{4mm}}
\pgfpathlineto{\pgfpointpolar{144}{4mm}}
\pgfpathlineto{\pgfpointpolar{288}{4mm}}
\pgfpathlineto{\pgfpointpolar{72}{4mm}}
\pgfpathlineto{\pgfpointpolar{216}{4mm}}
\pgfpathclose%
\pgfusepath{fill}
}

%%

%\tikzset{mystripes dist/.initial=0.25}
\tikzset{mystripes dist/.initial=1.75}
\pgfdeclarepatternformonly[/tikz/mystripes dist]{mystripes}
{\pgfpointorigin}{\pgfpoint{1cm}{1cm}}
{\pgfpoint{1cm}{1cm}}
{
  \foreach \x in {0,\pgfkeysvalueof{/tikz/mystripes dist},...,1}{
    \pgfmathsetmacro{\nx}{1-\x}
    \tikz[overlay,line width=0pt]\draw[decoration = {random steps,segment length = 0.5mm, amplitude = 0.15pt}, decorate] (\x, 0) -- ++(\nx,\nx);
    \tikz[overlay,line width=0pt]\draw[decoration = {random steps, segment length = 0.5mm, amplitude = 0.15pt}, decorate] (0, \x) -- ++(\nx,\nx);
  }
}

%%

%%%%%%%%%%%%%%%%%%%%%%%%%%%%%%%%%%%%%%%%%%%%%%%%%%%%%%%%%%%%%
%
% https://tex.stackexchange.com/questions/29769/colored-box-in-new-environment

\newsavebox{\selvestebox}
\newenvironment{colbox}[1]
  {\newcommand\colboxcolor{#1}%
   \begin{lrbox}{\selvestebox}%
   \begin{minipage}{\dimexpr\columnwidth-2\fboxsep\relax}}
  {\end{minipage}\end{lrbox}%
   %\begin{center}
   \colorbox{\colboxcolor}{\usebox{\selvestebox}}
   %\end{center}
   }




%%%%%%%%%%%%%%%%%%%%%%%%%%%%%%%%%%%%%%%%%%%%%%%%%%%%%%%%%%%%%
% https://tex.stackexchange.com/questions/443466/how-to-split-latex-page-horizontally-and-have-a-different-colour-background-to/443499
%
\newenvironment{advisepage}[2]
  {%
    \clearpage
    \AddToShipoutPictureBG*
      {%
        \AtPageUpperLeft
          {%
            \color{#1}%
            \raisebox{-.5\paperheight}{\rule{\paperwidth}{.5\paperheight}}%
          }%
        \color{#2}%
        \rule{\paperwidth}{.5\paperheight}%
      }%
  }
  {%
    \clearpage
  }
\newenvironment{advise}[2]
  {%
    \noindent
    \begin{minipage}[t][#1\textheight]{\textwidth}
    %\begin{colbox}{#2}
  }
  {%
    %\end{colbox}
    \end{minipage}%%
  }
\usepackage{duckuments}


%%%%%%%%%%%%%%%%%%%%%%%%

\newmdenv[align=center,innerlinewidth=.5pt, roundcorner=4pt,linecolor=gray!32,innerleftmargin=6pt,backgroundcolor=white,
innerrightmargin=6pt,innertopmargin=6pt,innerbottommargin=6pt]{logobox}