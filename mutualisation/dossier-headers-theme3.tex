
%%%%%%%%%%%%%%%%%%%%%%%%%%%%%%%%%%%%%%%%%%%%%%%%%%%%%%%%%%%%%%%%%%%%%% titlesec
% https://tex.stackexchange.com/questions/439166/beautiful-section-styles-with-polyglossia?noredirect=1&lq=1
%\newbox\TitleUnderlineTestBox
%\newcommand\TitleUnderline{\bgroup\markoverwith{%
%  \textcolor{titleblue}{\rule[-1.3ex]{2pt}{1.0pt}}}\ULon}


\newcommand\TitleUnderline{\bgroup\markoverwith{%
  {\rule[-1.3ex]{2pt}{1.0pt}}}\ULon}
\newcommand*\SectionNumberBox[2]
  {%
    \colorbox{#2}
      {%
        \makebox[2.5em][c]
          {%
            \color{white}%
            \strut
            \csname the#1\endcsname
          }%
      }%
    \TitleUnderline{\ \ \ }%
  }

\newcommand*\chapterlabel{}
\titleformat{\chapter}
  {%
    %\thispagestyle{chapterstart}
    \pagecolor{white}
    \bfseries\LARGE\raggedleft
    \color{general@best@maroon}
  }{~}
  {0pt}
  {\formatChapter}

  %\titlespacing*{\chapter}{0pt}{50pt}{-30pt}


\titleformat{\section}
  {\Large\bfseries\sffamily\color{general@best@navy}}
  {\SectionNumberBox{section}{general@best@navy}}
  {0pt}
  {\formatSection}
\titleformat{\subsection}
  {\large\bfseries\sffamily\color{general@best@red}}
  {\SectionNumberBox{subsection}{general@best@red}}
  {0pt}
  {\formatSubsection}


  %https://tex.stackexchange.com/questions/160320/customizing-chapter-style-with-tikz
  %http://www.texample.net/tikz/examples/fancy-chapter-headings/
  \newcommand{\formatChapter}[1]{%
    \begin{tikzpicture}[remember picture,overlay]
      \node[yshift=-3cm] at (current page.north west)
        {\begin{tikzpicture}[remember picture, overlay]
          \draw[fill=general@best@maroon] (0,0) rectangle
            (\paperwidth,3cm);
          \node[anchor=east,xshift=.9\paperwidth,rectangle,
                rounded corners=2pt,inner sep=11pt,
                fill=general@best@maroon!80]
                {\color{white}\chapterlabel#1};
        \end{tikzpicture}
        };
    \end{tikzpicture}
  }
  \newcommand{\formatSection}[1]{%
    \TitleUnderline{#1}
  }
  \newcommand{\formatSubsection}[1]{%
    \TitleUnderline{#1}
  }
%_______________________________________________________________________




%%%%%%%%%%%%%%%%%%%%%%%%%%%%%%%%%%%%%%%%%%%%%%%%%%%%%%%%%%%%%%%%%%%%%% sections


%\newcommand{\settoccolor}[1]{\color{#1}\def\@linkcolor{#1}}



\renewcommand{\mtifont}{\fontfamily{ugq}\selectfont}
\renewcommand{\mtcfont}{\fontfamily{ugq}\selectfont}
\renewcommand{\mtcSfont}{\fontfamily{ugq}\selectfont}
\renewcommand{\mtcSSfont}{\fontfamily{ugq}\selectfont}
\renewcommand{\mtcSSSfont}{\fontfamily{ugq}\selectfont}






%------------------------------------------------------------------------------ minitoc

% http://forum.mathematex.net/latex-f6/tocloft-minitoc-et-couleur-t9192.html


\renewcommand{\chaptermark}[1]{%
%\markboth{Module \thechapter\ -- #1}{}
\markboth{\textbf{Module : #1}}{}
}
\renewcommand{\sectionmark}[1]{%
\markright{\thesection\ -- #1}}

\renewcommand{\thepart}{%
  \ifcase\value{part}
  \or Premi\`ere\or Deuxi\`eme\or Troisi\`eme\or Quatri\`eme
  \or Cinqui\`eme\or Sixi\`eme\or Septi\`eme\or Huiti\`eme
  \or Neuvi\`eme\or Dixi\`eme\or Onzi\`eme\or Douzi\`eme
  \or Treizi\`eme\or Quatorzi\`eme\or Quinzi\`eme\or Seizi\`eme
  \or Dix-septi\`eme\or Dix-huiti\`eme\or Dix-neuvi\`eme
  \or Vingti\`eme
  \fi\space partie
}

\newcommand{\parttoccolor}{purple}
\newcommand{\chaptertoccolor}{purple}
\newcommand{\sectiontoccolor}{general@best@navy}
\newcommand{\subsectiontoccolor}{general@best@red}

% commande pour gérer la couleur dans la toc avec gestion d'hyperref
\makeatletter
\newcommand{\settoccolor}[1]{\color{#1}\def\@linkcolor{#1}}
\makeatother
% parts
\renewcommand{\cftpartfont}{\bfseries\large\settoccolor{\parttoccolor}}
\renewcommand{\cftpartafterpnum}{\settoccolor{\parttoccolor}\strut\par\nobreak\vspace{1pt}\hrule}
\let\cftpartpagefont\cftpartfont
% chapters
\renewcommand{\cftchappresnum}{Chapitre~}
\renewcommand{\cftchapleader}{\settoccolor{\chaptertoccolor}\leavevmode\cleaders\hbox to 1em {\hfil.\hfil}\hfill}
\renewcommand{\cftchapfont}{\settoccolor{\chaptertoccolor}\bfseries}
\let\cftchappagefont\cftchapfont
\settowidth{\cftchapnumwidth}{\cftchapfont Chapitre~XXX}
% sections
\renewcommand{\cftsecfont}{\settoccolor{\sectiontoccolor}}
\let\cftsecpagefont\cftsecfont
\renewcommand{\cftsecleader}{\settoccolor{\sectiontoccolor}\leavevmode\cleaders\hbox to 0.8em {\hfil.\hfil}\hfill}
\settowidth{\cftsecnumwidth}{\cftsecfont XXX.99}
% subsections
\renewcommand{\cftsubsecfont}{\settoccolor{\subsectiontoccolor}}
\let\cftsubsecpagefont\cftsubsecfont
\renewcommand{\cftsubsecleader}{\settoccolor{\subsectiontoccolor}\leavevmode\cleaders\hbox to 0.8em {\hfil.\hfil}\hfill}
\settowidth{\cftsecnumwidth}{\cftsubsecfont XXX.99}





%
\makeatletter
\def\contentsline#1#2#3#4{%
  \ifx\\#4\\%
    \csname l@#1\endcsname{#2}{#3}%
  \else
      \csname l@#1\endcsname{\hyper@linkstart{link}{#4}{#2}\hyper@linkend}{%
        \hyper@linkstart{link}{#4}{#3}\hyper@linkend
      }%
  \fi
}
\makeatother



%_______________________________________________________________________

