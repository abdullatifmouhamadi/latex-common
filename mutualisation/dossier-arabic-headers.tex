
% git clone http://git.robertalessi.net/arabluatex

\documentclass{report}
%\documentclass[ams,openany,10pt,presentation,utf8]{common/themes/dossiers/dossiers}

\usepackage[object=vectorian]{pgfornament} %%  http://altermundus.com/pages/tkz/ornament/index.html


%%%%%%%%%%%%%%%%%%%%%%%%%%%%%%%%%%%%%%%%%%%%%%%%%%%%%%%%%%%%%%%% MARGES 
% method 2
\usepackage{geometry}
\geometry{hmargin=1.5cm,vmargin=2.0cm}
%\geometry{hmargin=1.0cm,vmargin=1.5cm}


%%%%%%%%%%%%%%%%%%%%%%%%%%%%%%%%%%%%%%%%%%%%%%%%%%%%%%%%%%%%%%%% HEADERS/FOOTERS
\usepackage{fancyhdr}
\pagestyle{fancy}

\usepackage{amssymb}
\usepackage{tikz}
%\pagenumbering{arabic}



%\fancyhead[LE,RO]{\textsl{}}
%\fancyhead[LO,RE]{\textsl{}}

\fancyhead[LE,RO]{\textsl{\nouppercase{\rightmark}}}
\fancyhead[LO,RE]{\textsl{\nouppercase{\leftmark}}}
%\fancyhead[LE,RO]{\textsl{}}
%\fancyfoot[C]{\thepage}



%\fancyfoot[L]{CFAI Poitou-Charentes}
%\fancyfoot[C]{Physique Appliquée}
%\fancyfoot[R]{\thepage}
%\renewcommand{\headrulewidth}{0.4pt}
%\renewcommand{\footrulewidth}{0.4pt}

\fancyfoot[L]{\arb{'isl.Am}}
%\fancyfoot[C]{Electronique \& Communication}
%\fancyfoot[R]{\thepage}
\fancyfoot[R]{\today }
\renewcommand{\headrulewidth}{0.4pt}
%\renewcommand{\headrulewidth}{0.0pt}
\renewcommand{\footrulewidth}{0.4pt}



%%%%%%%%%%%%%%%%%%%%%%%%%%%%%%%%%%%%%%%%%%%%%%%%%%%%%%%%%%%%%%%% IMAGE FRAME 
\usepackage{efbox,graphicx}
%_______________________________________________________________

\usepackage{enumitem}




\usepackage[T1]{fontenc}
\usepackage{graphicx}
\usepackage{ifthen}

\usepackage[Lenny]{fncychap}





%%%%%%%%%%%%%%%%%%%%%%%%%%%%%%%%%%%%%%%%%%%%%%%%%%%%%%%%%%%%%%%% ARRAY 
\usepackage{array,multirow,makecell}
\renewcommand{\arraystretch}{1.2}

\usepackage{hhline}
\usepackage{makecell} % https://tex.stackexchange.com/questions/2441/how-to-add-a-forced-line-break-inside-a-table-cell


\usepackage{booktabs}


\usepackage{listings,multicol}
\usepackage{color}

\usepackage{xcolor, colortbl} %kleurtjes voor table

% print subsubsection
\setcounter{tocdepth}{5}
\setcounter{secnumdepth}{5}
% fin print subsubsection


\usepackage{minitoc}
\setcounter{minitocdepth}{3}
%\setcounter{minitocdepth}{2}
\setlength{\mtcindent}{24pt}
\renewcommand{\mtcfont}{\small\rm}
\renewcommand{\mtcSfont}{\small\bf}



\makeatletter
\let\th@plain\relax
\makeatother
%%%%%%%%%%%%%%%%%%%%%%%%%%%%%%%%%%%%%%%%%%%%%%%%%%%%%%%%%%%%%%%% PERSONNAL
%\usepackage{common/lesson/utils}
%\usepackage{common/lesson/headers}
\usepackage{common/lesson/pretty}
\usepackage{common/lesson/tables}





\usepackage[tuenc]{fontspec}

%\usepackage{polyglossia}
%\setdefaultlanguage{french}
%\setotherlanguage{arabic}
%\setmainfont{Amiri}


\usepackage{arabluatex,ulem}

%\newfontfamily\arabicfont[Script=Arabic]{Lateef}
%\newfontfamily\arabicfont[Script=Arabic]{Noto Naskh Arabic}
%\newfontfamily\arabicfont[Script=Arabic, Scale=2.0]{Amiri}
\newfontfamily\arabicfont[Script=Arabic, Scale=2.0]{Scheherazade}


\newcommand{\prth}[1]{\textfrench{)}#1\textfrench{(}}









\usepackage{hyperref}
%{\hypersetup{
%    colorlinks=true,
%    linkcolor=blue,
%    filecolor=blue,      
%    urlcolor=blue,
%    citecolor=blue,
    %pdfpagemode=FullScreen,
%}
\hypersetup{pdftex,colorlinks=true,allcolors=blue}
\usepackage{hypcap}


%%%%%%%%%%%%%%%%%%%%%%%%%%%%%%%%%%%%%%%%%%%%%%%%%%%%%%%%% remove chapter
\usepackage{titlesec}
%\titleformat{\chapter}[display]{\normalfont\bfseries}{}{0pt}{\Huge}
%\titleformat{\chapter}[display]{\normalfont}{}{0pt}{\Huge}
  %{\normalfont\bfseries}{}{0pt}{\Huge}
\usepackage{ulem}
\usepackage{tocloft}

\mtcselectlanguage{french}
% https://tex.stackexchange.com/questions/88238/remove-chapter-text-from-chapter-headings-and-have-chapter-number-on-the-same/88240
\renewcommand{\chaptermark}[1]{%
  \markboth{\thechapter. #1}{}}



%%%











\usepackage{csvsimple}


% https://tex.stackexchange.com/questions/178904/use-datatool-to-read-a-row-from-a-csv-file-then-use-the-variables-in-the-docume
\usepackage{datatool}
%\DTLsetdelimiter{|} 
\DTLsetseparator{|} % works
\usepackage{graphbox} %loads graphicx package

\newcommand\ara[1]{
    \begin{Arabic}
        #1
    \end{Arabic}
}







\title{\LARGE \textbf{MADINA ARABIC} \\ \normalsize (Apprentissage de la langue \& arabe) \\~\\ \Large Blended Learning}
\author{ABDULLATIF Mouhamadi}
%\author{}
\date{\today}


%\usepackage{lscape}
%\usepackage{pdflscape}
\usepackage{longtable}


\AtBeginDocument{
    \renewcommand{\labelitemi}{\textbullet}
    \renewcommand{\labelitemi}{$\circ$}
    \renewcommand{\labelitemii}{$\circ$}
    \renewcommand{\labelitemiii}{$\circ$}
    \renewcommand{\labelitemiv}{$\circ$}
}



\newcommand{\moduleMainHeader}[1]{%
\begin{center}

    \LARGE \textbf{DOSSIER PÉDAGOGIQUE} \\

    \textit{Ensemble d'éléments pour le suivi pédagogique} \\~

    \efbox[linecolor=black, linewidth=1pt,margin=8pt]{Promotion 2020-2022}
\end{center}
}





\usepackage{attrib}




% https://tex.stackexchange.com/questions/32711/totally-sweet-horizontal-rules-in-latex
\newcommand{\horizontall}{\noindent\rule{1.0\textwidth}{8pt}}


\newcommand{\horiza}{%
  \nointerlineskip \vspace{.5\baselineskip}\hspace{\fill}
  {\color{general@best@olive}
    \resizebox{0.75\linewidth}{2ex}
    {{%
    {\begin{tikzpicture}
    \node  (C) at (0,0) {};
    \node (D) at (9,0) {};
    \path (C) to [ornament=88] (D);
    \end{tikzpicture}}}}}%
    \hspace{\fill}
    \par\nointerlineskip \vspace{.5\baselineskip}
}

%%%%%%%%%%%%%% ARRAY VOCABULAIRE
\def\lexiqueacolor{general@best@aqua}
\newcommand{\lexiquea}[1]{%
  \begin{longtable}{R{.30\textwidth}L{.60\textwidth}}
    \arrayrulecolor{\lexiqueacolor}
    \hline 
    \rowcolor{\lexiqueacolor} \textbf{\textcolor{white}{arabe}} &  \textbf{\textcolor{white}{français}} \\\hline
    #1
    %\hline
  \end{longtable}
}

\newcommand{\lar}[3]{%
%#1&#2&#3\\\hline
\cellcolor{\lexiqueacolor!16}{#1}&#2\\\hline
}


%%%%%%%%%%%%%% ARRAY DIALOGUES
\def\dialogacolor{general@best@green}
\newcommand{\dialoga}[1]{%
  \begin{longtable}{R{.65\textwidth}L{.30\textwidth}}
    \arrayrulecolor{\dialogacolor}
    \hline 
    \rowcolor{\dialogacolor} \textbf{\textcolor{white}{arabe}} &  \textbf{\textcolor{white}{notes}} \\\hline
    #1
    %\hline
  \end{longtable}
}

\newcommand{\dar}[2]{%
%#1&#2&#3\\\hline
\cellcolor{\dialogacolor!16}{#1}&#2\\\hline
}


%%%%%%%%%%%%%


\newcommand{\rightPartie}[1]{%
\begin{flushright}
  \colorbox{general@best@maroon!100}{
      \textcolor{white!100}{
          \textbf{#1}
      }
  }
\end{flushright}
}


%%%%%%%%%%
\def\arbredacolor{general@best@red}
\newcommand{\reda}[1]{
  \textcolor{\arbredacolor}{#1}
}
\def\arbremacolor{general@best@fuchsia}
\newcommand{\rema}[1]{
  \textcolor{\arbremacolor}{#1}
}


