%\documentclass{report}
\documentclass[ams,openany,10pt,presentation,utf8]{common/themes/dossiers/dossiers}

%%%%%%%%%%%%%%%%%%%%%%%%%%%%%%%%%%%%%%%%%%%%%%%%%%%%%%%%%%%%%%%% MARGES 
% method 2
\usepackage{geometry}
\geometry{hmargin=1.5cm,vmargin=2.0cm}
%\geometry{hmargin=1.0cm,vmargin=1.5cm}


%%%%%%%%%%%%%%%%%%%%%%%%%%%%%%%%%%%%%%%%%%%%%%%%%%%%%%%%%%%%%%%%


 

%%%%%%%%%%%%%%%%%%%%%%%%%%%%%%%%%%%%%%%%%%%%%%%%%%%%%%%%%%%%%%%% TIMING
\usepackage{tikz-timing}
\usetikzlibrary{positioning,arrows.meta}
\usetikztiminglibrary[new={char=Q,reset char=R}]{counters}
\usetikztiminglibrary{nicetabs} 

%%%%%%%%%%%%%%%%%%%%%%%%%%%%%%%%%%%%%%%%%%%%%%%%%%%%%%%%%%%%%%%%




%%%%%%%%%%%%%%%%%%%%%%%%%%%%%%%%%%%%%%%%%%%%%%%%%%%%%%%%%%%%%%%% HEADERS/FOOTERS
\usepackage{fancyhdr}
\pagestyle{fancy}



%\fancyhead[LE,RO]{\textsl{}}
%\fancyhead[LO,RE]{\textsl{}}

\fancyhead[LE,RO]{\textsl{\rightmark}}
\fancyhead[LO,RE]{\textsl{\leftmark}}
%\fancyhead[LE,RO]{\textsl{}}
\fancyfoot[C]{\thepage}


%\fancyfoot[L]{CFAI Poitou-Charentes}
%\fancyfoot[C]{Physique Appliquée}
%\fancyfoot[R]{\thepage}
%\renewcommand{\headrulewidth}{0.4pt}
%\renewcommand{\footrulewidth}{0.4pt}

\fancyfoot[L]{CFAI Poitou-Charentes}
%\fancyfoot[C]{Electronique \& Communication}
%\fancyfoot[R]{\thepage}
\fancyfoot[R]{\today }
\renewcommand{\headrulewidth}{0.4pt}
%\renewcommand{\headrulewidth}{0.0pt}
\renewcommand{\footrulewidth}{0.4pt}

%%%%%%%%%%%%%%%%%%%%%%%%%%%%%%%%%%%%%%%%%%%%%%%%%%%%%%%%%%%%%%%%


%%%%%%%%%%%%%%%%%%%%%%%%%%%%%%%%%%%%%%%%%%%%%%%%%%%%%%%%%%%%%%%% IMAGE FRAME 
\usepackage{efbox,graphicx}
%_______________________________________________________________



\usepackage{enumitem}
%%%%%%%%%%%%%%%%%%%%%%%%%%%%%%%%%%%%%%%%%%%%%%%%%%%%%%%%%%%%%%%%

\usepackage{amsfonts,amsmath,amssymb,amsthm}
\usepackage{aligned-overset}

\usepackage{mathptmx}
\usepackage{relsize}
\usepackage{physics}

\usepackage[T1]{fontenc}
\usepackage{graphicx}
\usepackage{ifthen}

\usepackage[Lenny]{fncychap}


\usepackage{listings,multicol}
\usepackage{color}

% print subsubsection
\setcounter{tocdepth}{5}
\setcounter{secnumdepth}{5}
% fin print subsubsection


\usepackage{minitoc}
\setcounter{minitocdepth}{3}
%\setcounter{minitocdepth}{2}
\setlength{\mtcindent}{24pt}
\renewcommand{\mtcfont}{\small\rm}
\renewcommand{\mtcSfont}{\small\bf}

\usepackage{hyperref}
\hypersetup{pdftex,colorlinks=true,allcolors=blue}
\usepackage{hypcap}


\usepackage{type1cm}
\usepackage{lettrine}
\usepackage{scrextend}
\usepackage{blindtext}
%\usepackage[svgnames, x11names]{xcolor}



\usepackage[utf8]{inputenc}
\usepackage{lmodern,textcomp}

\usepackage[french]{babel}
\mtcselectlanguage{french}

\usepackage{multido}
\RequirePackage{framed}
\usepackage{fancyhdr}
\usepackage{fancybox}
\usepackage{fancybox,xcolor}
\usepackage{blindtext}
\usepackage{xcolor, colortbl} %kleurtjes voor table

\usepackage{hhline}
\usepackage{makecell} % https://tex.stackexchange.com/questions/2441/how-to-add-a-forced-line-break-inside-a-table-cell



\usepackage{answers}

%\usepackage{mdframed}

\usepackage[framemethod=TikZ]{mdframed}

\usepackage{pstricks,pst-grad}

% ntheorem
\makeatletter
\let\th@plain\relax
\makeatother
%\usepackage[thref]{ntheorem}
%\usepackage[amsmath,thref]{ntheorem}
\usepackage{empheq}
\usepackage{thmtools}
\usepackage{subfig}
\usepackage{hyperref}%
\usepackage{systeme,mathtools}


%%%%%%%%%%%%%%%%%%%%%%%%%%%%%%%%%%%%%%%%%%%%%%%%%%%%%%%%%%%%%%%% CIRCUITS LOGIQUE
\usetikzlibrary{babel} %%% <--- Don't forget
%\usepackage[siunitx, straightvoltages]{circuitikz}
\usepackage[european,straightvoltages]{circuitikz}
\usetikzlibrary{circuits.logic.IEC}
\usetikzlibrary{shapes.gates.logic.IEC}
\usetikzlibrary{circuits.logic.US,circuits.logic.IEC, positioning}
\usetikzlibrary{arrows, shapes.gates.logic.US, calc}
\usepackage{pgfplots}
\usetikzlibrary{angles,quotes}

\usetikzlibrary{decorations.markings}
\usetikzlibrary{arrows, positioning}



%%% params
\ctikzset{v/.append style={/tikz/european voltages}}
%_______________________________________________________________



%%%%%%%%%%%%%%%%%%%%%%%%%%%%%%%%%%%%%%%%%%%%%%%%%%%%%%%%%%%%%%%% ARRAY 
\usepackage{array,multirow,makecell}
\renewcommand{\arraystretch}{1.2}
%\setcellgapes{1pt}
%\makegapedcells
%\newcolumntype{R}[1]{>{\raggedleft\arraybackslash }b{#1}}
%\newcolumntype{L}[1]{>{\raggedright\arraybackslash }b{#1}}
%\newcolumntype{C}[1]{>{\centering\arraybackslash }b{#1}}
%%%%%%%%%%%%%%%%%%%%%%%%%%%%%%%%%%%%%%%%%%%%%%%%%%%%%%%%%%%%%%%%

%%%%%%%%%%%%%%%%%%%%%%%%%%%%%%%%%%%%%%%%%%%%%%%%%%%%%%%%%%%%%%%% PERSONNAL
\usepackage{common/lesson/utils}
\usepackage{common/lesson/headers}
\usepackage{common/lesson/tables}
%\usepackage{common/lesson/blocks}
\usepackage{common/lesson/theorems}
%\usepackage{common/lesson/tests}
%\usepackage{common/lesson/ccf}
%\usepackage{common/images/nodes}

\usepackage{common/lesson/dossier}
\usepackage{common/lesson/pretty}

\usepackage{common/images/nodes}


\definecolor{Gray}{gray}{0.9}

%%%%%%%%%%%%%%%%%%%%%%%%%%%%%%%%%%%%%%%%%%%%%%%%%%%%%%%%%%%%%%%%

% moi-même
\renewcommand{\thesection}{\arabic{section}} % Remove chapter number to sections

\usepackage{chngcntr}
\usepackage{xhfill}


% Circle
\usepackage{pict2e,picture}
\newsavebox\CBox
\newlength\CLength
\def\competence#1{\sbox\CBox{#1}%
  \ifdim\wd\CBox>\ht\CBox \CLength=\wd\CBox\else\CLength=\ht\CBox\fi
    \makebox[1.1\CLength]{\makebox(0,1.1\CLength){\put(0,0){\circle{1.0\CLength}}\put(0,0){\circle{0.9\CLength}}}%
    \makebox(0,1.1\CLength){\put(-.3\wd\CBox,0){\tiny\textbf{#1}}}}}
% End circle




%%%%%%%%%%%%%%%%%%%%%%%%%%%%%%%%%%%%%%%%%%%%%%%%%%%%%%%%% remove chapter
%\usepackage{titlesec}
\usepackage{xcolor}
\usepackage{titlesec}
%\usepackage[explicit]{titlesec}
\usepackage{ulem}
%\titleformat{\chapter}[display]{\normalfont\bfseries}{}{0pt}{\Huge}
  %{\normalfont\bfseries}{}{0pt}{\Huge}




%\definecolor{etude@color}		{RGB}{133, 20, 75}

\newenvironment{etude}
{
  \itemThemeBasic{etude@color}{\fontfamily{ugq}\large\selectfont}
}
{
~
}








% http://forum.mathematex.net/latex-f6/tocloft-minitoc-et-couleur-t9192.html
%\usepackage{tocloft}

% https://tex.stackexchange.com/questions/336617/command-clofdepth-already-defined-when-using-packages-subfigure-and-tocloft
\usepackage[subfigure]{tocloft}





%%% tmp listings
\definecolor{mygreen}{rgb}{0,0.6,0}
\definecolor{mygray}{rgb}{0.5,0.5,0.5}
\definecolor{mymauve}{rgb}{0.58,0,0.82}

\lstset{ 
  backgroundcolor=\color{white},   % choose the background color; you must add \usepackage{color} or \usepackage{xcolor}; should come as last argument
  basicstyle=\footnotesize,        % the size of the fonts that are used for the code
  breakatwhitespace=false,         % sets if automatic breaks should only happen at whitespace
  breaklines=true,                 % sets automatic line breaking
  captionpos=b,                    % sets the caption-position to bottom
  commentstyle=\color{mygreen},    % comment style
  deletekeywords={...},            % if you want to delete keywords from the given language
  escapeinside={\%*}{*)},          % if you want to add LaTeX within your code
  extendedchars=true,              % lets you use non-ASCII characters; for 8-bits encodings only, does not work with UTF-8
  firstnumber=1,                % start line enumeration with line 1000
  frame=single,	                   % adds a frame around the code
  keepspaces=true,                 % keeps spaces in text, useful for keeping indentation of code (possibly needs columns=flexible)
  keywordstyle=\color{blue},       % keyword style
  language=C,                 % the language of the code
  morekeywords={*,...},            % if you want to add more keywords to the set
  numbers=left,                    % where to put the line-numbers; possible values are (none, left, right)
  numbersep=5pt,                   % how far the line-numbers are from the code
  numberstyle=\tiny\color{mygray}, % the style that is used for the line-numbers
  rulecolor=\color{black},         % if not set, the frame-color may be changed on line-breaks within not-black text (e.g. comments (green here))
  showspaces=false,                % show spaces everywhere adding particular underscores; it overrides 'showstringspaces'
  showstringspaces=false,          % underline spaces within strings only
  showtabs=false,                  % show tabs within strings adding particular underscores
  stepnumber=2,                    % the step between two line-numbers. If it's 1, each line will be numbered
  stringstyle=\color{mymauve},     % string literal style
  tabsize=2,	                   % sets default tabsize to 2 spaces
  %title=\lstname                   % show the filename of files included with \lstinputlisting; also try caption instead of title
}


\lstdefinestyle{customc}{
  belowcaptionskip=1\baselineskip,
  breaklines=true,
  %frame=L,
  xleftmargin=\parindent,
  language=C,
  showstringspaces=false,
  basicstyle=\normalsize\ttfamily,
  keywordstyle=\bfseries\color{green!40!black},
  commentstyle=\itshape\color{purple!40!black},
  identifierstyle=\color{blue},
  stringstyle=\color{orange},
}

\lstdefinestyle{customasm}{
  belowcaptionskip=1\baselineskip,
  frame=L,
  xleftmargin=\parindent,
  language=[x86masm]Assembler,
  basicstyle=\footnotesize\ttfamily,
  commentstyle=\itshape\color{purple!40!black},
}



\lstset{escapechar=@,style=customc}

\lstset{literate=
  {á}{{\'a}}1 {é}{{\'e}}1 {í}{{\'i}}1 {ó}{{\'o}}1 {ú}{{\'u}}1
  {Á}{{\'A}}1 {É}{{\'E}}1 {Í}{{\'I}}1 {Ó}{{\'O}}1 {Ú}{{\'U}}1
  {à}{{\`a}}1 {è}{{\`e}}1 {ì}{{\`i}}1 {ò}{{\`o}}1 {ù}{{\`u}}1
  {À}{{\`A}}1 {È}{{\'E}}1 {Ì}{{\`I}}1 {Ò}{{\`O}}1 {Ù}{{\`U}}1
  {ä}{{\"a}}1 {ë}{{\"e}}1 {ï}{{\"i}}1 {ö}{{\"o}}1 {ü}{{\"u}}1
  {Ä}{{\"A}}1 {Ë}{{\"E}}1 {Ï}{{\"I}}1 {Ö}{{\"O}}1 {Ü}{{\"U}}1
  {â}{{\^a}}1 {ê}{{\^e}}1 {î}{{\^i}}1 {ô}{{\^o}}1 {û}{{\^u}}1
  {Â}{{\^A}}1 {Ê}{{\^E}}1 {Î}{{\^I}}1 {Ô}{{\^O}}1 {Û}{{\^U}}1
  {Ã}{{\~A}}1 {ã}{{\~a}}1 {Õ}{{\~O}}1 {õ}{{\~o}}1
  {œ}{{\oe}}1 {Œ}{{\OE}}1 {æ}{{\ae}}1 {Æ}{{\AE}}1 {ß}{{\ss}}1
  {ű}{{\H{u}}}1 {Ű}{{\H{U}}}1 {ő}{{\H{o}}}1 {Ő}{{\H{O}}}1
  {ç}{{\c c}}1 {Ç}{{\c C}}1 {ø}{{\o}}1 {å}{{\r a}}1 {Å}{{\r A}}1
  {€}{{\euro}}1 {£}{{\pounds}}1 {«}{{\guillemotleft}}1
  {»}{{\guillemotright}}1 {ñ}{{\~n}}1 {Ñ}{{\~N}}1 {¿}{{?`}}1
}

%%


% https://math-linux.com/latex-4/faq/latex-faq/article/comment-ecrire-un-algorithme-ou-du-pseudocode-en-latex-usepackage-algorithm-usepackage-algorithmic
\usepackage{algorithm}
\usepackage{algorithmic}


\usepackage{csvsimple}


% https://tex.stackexchange.com/questions/178904/use-datatool-to-read-a-row-from-a-csv-file-then-use-the-variables-in-the-docume
\usepackage{datatool}
%\DTLsetdelimiter{";"} ; ne marche pas je sais pas pourquoi



