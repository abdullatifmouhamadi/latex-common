
% dossier-headers-business-contract.tex
% https://www.uweziegenhagen.de/?p=2224
% https://www2.icp.uni-stuttgart.de/~icp/mediawiki/images/5/51/Latex-template.tex


\RequirePackage[l2tabu,orthodox]{nag} % turn on warnings because of bad style
\documentclass[a4paper,11pt,bibtotoc]{scrartcl}
%\usepackage{common/mutualisation/default-packages}
\usepackage{common/business/business-invoice}



\usepackage[french]{babel}
%\usepackage[a5paper,top=2cm, bottom=2cm, left = 2.5cm, right=2cm]{geometry}

\usepackage{geometry}
\geometry{hmargin=1.0cm,vmargin=2.5cm}


\usepackage[utf8]{inputenc}

\usepackage[T1]{fontenc}        % Tries to use Postscript Type 1 Fonts for better rendering
\usepackage{lmodern}            % Provides the Latin Modern Font which offers more glyphs than the default Computer Modern
\usepackage[intlimits]{amsmath} % Provides all mathematical commands

\usepackage{hyperref}           % Provides clickable links in the PDF-document for \ref
\usepackage{grffile}            % Allow you to include images (like graphicx). Usage: \includegraphics{path/to/file}

% Allows to set units
\usepackage[ugly]{units}        % Allows you to type units with correct spacing and font style. Usage: $\unit[100]{m}$ or $\unitfrac[100]{m}{s}$

% Additional packages
\usepackage{url}                % Lets you typeset urls. Usage: \url{http://...}
\usepackage{breakurl}           % Enables linebreaks for urls
\usepackage{xspace}             % Use \xpsace in macros to automatically insert space based on context. Usage: \newcommand{\es}{ESPResSo\xspace}
\usepackage{xcolor}             % Obviously colors. Usage: \color{red} Red text
\usepackage{booktabs}           % Nice rules for tables. Usage \begin{tabular}\toprule ... \midrule ... \bottomrule


\usepackage{multicol}
\setlength{\columnsep}{0.5cm}

\usepackage{tikz}
\usepackage{tikzpagenodes}


\usepackage{enumitem}
\setlist[itemize]{align=parleft,left=0pt..1em}
\setlist[enumerate]{align=parleft,left=0pt..1em}


\usepackage[framemethod=tikz]{mdframed}
\usepackage{hhline}


\usepackage{hyperref}
\hypersetup{
    colorlinks=true,
    linkcolor=blue,
    filecolor=magenta,      
    urlcolor=cyan,
}

\urlstyle{same}


% ------------------------------------------------ certifications

% https://bofip.impots.gouv.fr/bofip/10692-PGP.html/identifiant=BOI-LETTRE-000242-20201230


\newcommand{\poscertifactionvoleta}[7]{
    \Large
    \textbf{\normalsize Volet 1 : Partie à remplir par l'éditeur ou intégrateur du logiciel ou du système de caisse}\\

    Je soussigné, #1, représentant légal de la société #2, 
    éditeur du logiciel de caisse #3, atteste que ce logiciel 
    OU les fonctionnalités de caisse de ce logiciel, 
    mis sur le marché à compter du #4, 
    dans sa version n°#5, satisfait OU satisfont aux 
    conditions d’inaltérabilité, de sécurisation, de conservation et d’archivage 
    des données en vue du contrôle de l’administration fiscale, prévues au 3° bis 
    du I de l’\href{https://www.legifrance.gouv.fr/codes/article\_lc/LEGIARTI000036432356/2018-01-01}{article 286 du code général des impôts}.\\


    J'atteste que la dernière version majeure de ce logiciel est identifiée 
    avec la racine suivante : #5 et que les versions mineures développées ultérieurement 
    à cette version majeure sont ou seront identifiées par les subdivisions suivantes de 
    cette racine : #5.x.x. Je m'engage à ce que ces subdivisions ne soient utilisées 
    par #2 de l'éditeur que pour l'identification des versions mineures 
    ultérieures, à l'exclusion de toute version majeure. Les versions majeures et mineures 
    du logiciel ou système s'entendent au sens du 
    \href{https://bofip.impots.gouv.fr/bofip/10691-PGP.html/identifiant=BOI-TVA-DECLA-30-10-30-20201230#340\_0497}{III-A § 340 du BOI-TVA-DECLA-30-10-30}\\


    Fait à #6,\hspace{5cm} Le #7,\\

    Signature du représentant légal de l’éditeur du logiciel de caisse :\\~\\


    \normalsize\textbf{Il est rappelé que l’établissement d’une fausse attestation est un 
    délit pénal passible de 3 ans d’emprisonnement et de 45 000 € 
    d’amende (\href{https://www.legifrance.gouv.fr/codes/article\_lc/LEGIARTI000006418753/2002-01-01}{code pénal, art. 441-1}). 
    L'usage d'une fausse attestation est passible des mêmes peines.}

}


\newcommand{\poscertifactionvoletb}[6]{
    \Large
    \textbf{\normalsize Volet 2 : Partie à remplir par l'entreprise qui utilise le logiciel  ou le système de caisse}\\

    Je soussigné, #1, représentant légal de la  société  #2, 
    certifie avoir acquis ou téléchargé le #3, auprès de #4, 
    le logiciel de caisse mentionné au volet 1 de cette attestation.\\

    J'atteste utiliser ce logiciel de caisse pour enregistrer les 
    règlements de mes clients particuliers, conformément à la réglementation 
    fiscale en vigueur, depuis le #3.\\

    Fait à #5,\hspace{5cm} Le #6,\\

    Signature du représentant légal :\\~\\

    \normalsize\textbf{Il est rappelé que l’établissement d’une fausse attestation est un 
    délit pénal passible de 3 ans d’emprisonnement et de 45 000 € d’amende 
    (\href{https://www.legifrance.gouv.fr/codes/article\_lc/LEGIARTI000006418753/2002-01-01}{code pénal, art. 441-1}). 
    L'usage d'une fausse attestation est passible des mêmes peines.}\\

    \normalsize\textbf{Les volets 1 et 2 de cette attestation doivent être présentés à 
    l’administration fiscale en cas de contrôle. 
    Elle n’a de valeur que si son volet 2 est dûment complété et signé par 
    l’entreprise utilisatrice du logiciel.}

}
