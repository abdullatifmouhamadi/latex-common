%%%%%%%%%%%%% \breakbox

\def\typebox#1{%
\expandafter\def\csname spacebefore\endcsname{\expandafter\csname spacebefore#1\endcsname}%
\expandafter\def\csname stylebox\endcsname{\expandafter\csname#1@box\endcsname}%
\expandafter\def\csname titlefont\endcsname{\expandafter\csname#1title@font\endcsname}%
\expandafter\def\csname pluriel\endcsname{\expandafter\csname pluriel@#1\endcsname}%
\ifnum\thetype@box=0\setlength{\spacebottom}{\expandafter\csname spacebottom#1\endcsname}\fi
\expandafter\def\csname fillbg\endcsname{#1@bg@color}
\expandafter\def\csname titlecolor\endcsname{#1title@color}
\expandafter\def\csname titlebgcolor\endcsname{#1title@bg@color}
}
\newcounter{break@box}
\newcounter{type@box}
\newcommand*{\breakbox}
{%
	
		\ifnum\thetype@box=0 % pour "rem" ou "meth" ou "new"
		\end{minipage}
			\end{lrbox}
			\vspace*{\spacebefore}
			\stepcounter{break@box}
			\begin{tikzpicture}
			\node (texte) {\usebox{\stylebox}\vspace*{\spacebottom}};
			\fill[fill=shadow@color] ($(texte.south west)+(1pt,-1pt)$) -- ($(texte.north west)+(1pt,-1pt)$) -- ($(texte.north east)+(1pt,-1pt)$) -- ($(texte.south east)+(0,6.5pt)+(1pt,-2pt)$) .. controls ($(texte.south east)+(-2pt,2pt)+(1pt,-1pt)$) and ($(texte.south east)+(-3pt,1pt)+(1pt,-1pt)$) .. ($(texte.south east)+(-6.5pt,0)+(1pt,-1pt)$) -- cycle;
			\fill[fill=\fillbg] (texte.south west) -- (texte.north west) -- (texte.north east) -- ($(texte.south east)+(0,5pt)$) -- ($(texte.south east)+(-5pt,0)$) -- cycle;
			\fill[fill=shadow@color!50!black] ($(texte.south east)+(0,5pt)$) .. controls ($(texte.south east)+(-2pt,2pt)$) and ($(texte.south east)+(-3pt,1pt)$) .. ($(texte.south east)+(-5pt,0)$) .. controls ($(texte.south east)+(-4.5pt,2.5pt)$) .. ($(texte.south east)+(-5pt,5pt)$) .. controls ($(texte.south east)+(-2.5pt,4.5pt)$) .. ($(texte.south east)+(0,5pt)$);
			\node at(texte) {\usebox{\stylebox}};
			\node[rounded corners=3pt,fill=\fillbg!75,draw=\titlecolor] at ($(texte.south east)-(1,0)$) {...};
			\node[above right,fill=\titlebgcolor,text=\titlecolor] at (texte.north west) {\titlefont \titletext\pluriel};
			\end{tikzpicture}\newline
			\begin{lrbox}{\stylebox}
			\begin{minipage}{\linewidth}
		\fi
		\ifnum\thetype@box=1 % pour "def" ou "prop"
			\end{minipage}
			\end{lrbox}
			\vspace*{\spacebefore}
			\stepcounter{break@box}
			\begin{tikzpicture}
			\node[fill=\fillbg] (texte) {\usebox{\stylebox}};
			%\draw[very thick,draw=\titlebgcolor] (texte.north west) -- (texte.south west);
			\node[rounded corners=3pt,fill=\fillbg!75,draw=\titlebgcolor] at ($(texte.south east)-(1,0)$) {...};
			%\node[fill=deftitle@bg@color,text=deftitle@color,below left,draw=deftitle@bg@color,very thick,text width={\dimexpr\marginleft-2mm},align=center] at (texte.north west) {\deftitle@font \titletext\pluriel@def \ifnum\thebreak@box>0\par(suite)\fi};
			\node[fill=\titlebgcolor,text=\titlecolor,below left,draw=\titlebgcolor,very thick,text width={\dimexpr\marginleft-2mm},align=center] at ($(texte.north west)+(1pt,0)$) {\titlefont \titletext\pluriel};
			\end{tikzpicture}\newline
			\begin{lrbox}{\stylebox}
			\begin{minipage}{\dimexpr\linewidth-\marginleft}
		\fi
		\ifnum\thetype@box=2 % pour "dem"
			\vskip-1em\hfill{\demtitle@font(...)}
			\end{minipage}
			\end{lrbox}
			\vspace*{\spacebefore}
			\begin{tikzpicture}
			\node[fill=dem@bg@color] (texte) {\usebox{\dem@box}};
			\ifnum\indic=1
				\draw[thick,draw=dem@border@color,decorate, decoration={random steps, segment length=3pt,amplitude=1pt}] (texte.north west) -- (texte.south west);
			\else
				\ifnum\indic=2
					\draw[thick,draw=dem@border@color,decorate, decoration={random steps, segment length=3pt,amplitude=1pt}] (texte.north west) -- (texte.south west) -- (texte.south east);
				\else
					\ifnum\indic=3
						\draw[thick,draw=dem@border@color,decorate, decoration={random steps, segment length=3pt,amplitude=1pt}] (texte.north west) -- (texte.south west);
						\draw[thick,draw=dem@border@color,decorate, decoration={random steps, segment length=3pt,amplitude=1pt}] (texte.north east) -- (texte.south east);
					\else
						\ifnum\indic=4
							\draw[thick,draw=dem@border@color,decorate, decoration={random steps, segment length=3pt,amplitude=1pt}] (texte.north east) -- (texte.north west) -- (texte.south west);
						\else
							\ifnum\indic=5
								\draw[thick,draw=dem@border@color,decorate, decoration={random steps, segment length=3pt,amplitude=1pt}] (texte.north west) -- (texte.south west) -- (texte.south east) -- (texte.north east) -- cycle;
							\fi
						\fi
					\fi
				\fi
			\fi
			\node[below left,minimum width=0.5\marginleft] at (texte.north west) {\phantom{-}};
			\end{tikzpicture}\newline
			\begin{lrbox}{\dem@box}
			\begin{minipage}{\dimexpr\linewidth-0.5\marginleft}
			{\demtitle@font \titletext~(suite)}\par\medskip%
		\fi
}

%%%%%%%%%%%%% Définition d'une nouvelle Box

\newsavebox{\new@box}
\newcommand{\DefineNewBoxLikeRem}[5] % #1 : nom ; #2 : titre ; #3 : couleur principale ; #4 : couleur texte des items ; #5 : fonte des items
{%
	\newenvironment{#1}
	{%
		\setcounter{break@box}{0}
		\setcounter{type@box}{0}
		% Redéfinition de \breakbox
		\renewcommand*{\breakbox}
		{%
			\end{minipage}
				\end{lrbox}
				\vspace*{\spacebeforenew}
				\stepcounter{break@box}
				\begin{tikzpicture}
				\node (texte) {\usebox{\new@box}\vspace*{\spacebottomnew}};
				\fill[fill=shadow@color] ($(texte.south west)+(1pt,-1pt)$) -- ($(texte.north west)+(1pt,-1pt)$) -- ($(texte.north east)+(1pt,-1pt)$) -- ($(texte.south east)+(0,6.5pt)+(1pt,-2pt)$) .. controls ($(texte.south east)+(-2pt,2pt)+(1pt,-1pt)$) and ($(texte.south east)+(-3pt,1pt)+(1pt,-1pt)$) .. ($(texte.south east)+(-6.5pt,0)+(1pt,-1pt)$) -- cycle;
				\fill[fill=#3!20] (texte.south west) -- (texte.north west) -- (texte.north east) -- ($(texte.south east)+(0,5pt)$) -- ($(texte.south east)+(-5pt,0)$) -- cycle;
				\fill[fill=shadow@color!50!black] ($(texte.south east)+(0,5pt)$) .. controls ($(texte.south east)+(-2pt,2pt)$) and ($(texte.south east)+(-3pt,1pt)$) .. ($(texte.south east)+(-5pt,0)$) .. controls ($(texte.south east)+(-4.5pt,2.5pt)$) .. ($(texte.south east)+(-5pt,5pt)$) .. controls ($(texte.south east)+(-2.5pt,4.5pt)$) .. ($(texte.south east)+(0,5pt)$);
				\node at(texte) {\usebox{\new@box}};
				\node[rounded corners=3pt,fill=#3!20,draw=#3,text=#4] at ($(texte.south east)-(1,0)$) {...};
				\node[above right,fill=#3,text=#4] at (texte.north west) {\fontfamily{#5}\selectfont#2};
				\end{tikzpicture}\newline
				\begin{lrbox}{\new@box}
				\begin{minipage}{\linewidth}
		}
		\def\titletext{#2} % titre de la box
		\itemclass{#3}{\fontfamily{#5}\selectfont}
		\begin{lrbox}{\new@box}
		\begin{minipage}{\linewidth}
	}
	{%
		\end{minipage}
		\end{lrbox}
		\vspace*{\spacebeforenew}
		\begin{tikzpicture}
		\node (texte) {\usebox{\new@box}\vspace*{\spacebottomnew}};
		\fill[fill=shadow@color] ($(texte.south west)+(1pt,-1pt)$) -- ($(texte.north west)+(1pt,-1pt)$) -- ($(texte.north east)+(1pt,-1pt)$) -- ($(texte.south east)+(0,6.5pt)+(1pt,-2pt)$) .. controls ($(texte.south east)+(-2pt,2pt)+(1pt,-1pt)$) and ($(texte.south east)+(-3pt,1pt)+(1pt,-1pt)$) .. ($(texte.south east)+(-6.5pt,0)+(1pt,-1pt)$) -- cycle;
		\fill[fill=#3!20] (texte.south west) -- (texte.north west) -- (texte.north east) -- ($(texte.south east)+(0,5pt)$) -- ($(texte.south east)+(-5pt,0)$) -- cycle;
		\fill[fill=shadow@color!50!black] ($(texte.south east)+(0,5pt)$) .. controls ($(texte.south east)+(-2pt,2pt)$) and ($(texte.south east)+(-3pt,1pt)$) .. ($(texte.south east)+(-5pt,0)$) .. controls ($(texte.south east)+(-4.5pt,2.5pt)$) .. ($(texte.south east)+(-5pt,5pt)$) .. controls ($(texte.south east)+(-2.5pt,4.5pt)$) .. ($(texte.south east)+(0,5pt)$);
		\node at(texte) {\usebox{\new@box}};
		\node[above right,fill=#3,text=#4] at (texte.north west) {\fontfamily{#5}\selectfont#2\ifnum\thebreak@box>0 ~(suite)\fi};
		\end{tikzpicture}
	}
%
}	

\newcommand{\DefineNewBoxLikeDef}[5] % #1 : nom ; #2 : titre ; #3 : couleur principale ; #4 : couleur texte des items ; #5 : fonte des items
{%
	\newenvironment{#1}
	{%
		\setcounter{break@box}{0}
		\setcounter{type@box}{1}
		\renewcommand*{\breakbox}
		{%
			\end{minipage}
			\end{lrbox}
			\vspace*{\spacebeforenew}
			\stepcounter{break@box}
			\begin{tikzpicture}
			\node[fill=#3!20] (texte) {\usebox{\new@box}};
			\node[rounded corners=3pt,fill=#3!20,draw=#3] at ($(texte.south east)-(1,0)$) {...};
			\node[fill=#3,text=#4,below left,draw=#3,very thick,text width={\dimexpr\marginleft-2mm},align=center] at ($(texte.north west)+(1pt,0)$) {\fontfamily{#5}\selectfont#2};
			\end{tikzpicture}\newline
			\begin{lrbox}{\new@box}
			\begin{minipage}{\dimexpr\linewidth-\marginleft}
		}
		\def\titletext{#2}
		\itemclass{#3}{\fontfamily{#5}\selectfont}
		\begin{lrbox}{\new@box}
		\begin{minipage}{\dimexpr\linewidth-\marginleft}
	}
	{%
		\end{minipage}
		\end{lrbox}
		\vspace*{\spacebeforenew}
		\begin{tikzpicture}
		\node[fill=#3!20] (texte) {\usebox{\new@box}};
		\node[fill=#3,text=#4,below left,draw=#3,very thick,text width={\dimexpr\marginleft-2mm},align=center] at ($(texte.north west)+(1pt,0)$) {\fontfamily{#5}\selectfont#2\ifnum\thebreak@box>0\par(suite)\fi};
		\end{tikzpicture}
	}
}

%%%%%%%%%%%%% Environnement "remarque"

\newsavebox{\rem@box}
\newenvironment{remarque}[1][]
{%
\setcounter{break@box}{0}
\setcounter{type@box}{0}
\typebox{rem}
\def\titletext{Remarque}
\ifx#1\@empty\xdef\pluriel@rem{}\else\xdef\pluriel@rem{#1}\fi
\itemclass{remtitle@bg@color}{\remtitle@font}
\begin{lrbox}{\rem@box}
\begin{minipage}{\linewidth}
}
{%
\end{minipage}
\end{lrbox}
\vspace*{\spacebeforerem}
\begin{tikzpicture}
\node (texte) {\usebox{\rem@box}\vspace*{\spacebottomrem}};
\fill[fill=shadow@color] ($(texte.south west)+(1pt,-1pt)$) -- ($(texte.north west)+(1pt,-1pt)$) -- ($(texte.north east)+(1pt,-1pt)$) -- ($(texte.south east)+(0,6.5pt)+(1pt,-2pt)$) .. controls ($(texte.south east)+(-2pt,2pt)+(1pt,-1pt)$) and ($(texte.south east)+(-3pt,1pt)+(1pt,-1pt)$) .. ($(texte.south east)+(-6.5pt,0)+(1pt,-1pt)$) -- cycle;
\fill[fill=rem@bg@color] (texte.south west) -- (texte.north west) -- (texte.north east) -- ($(texte.south east)+(0,5pt)$) -- ($(texte.south east)+(-5pt,0)$) -- cycle;
\fill[fill=shadow@color!50!black] ($(texte.south east)+(0,5pt)$) .. controls ($(texte.south east)+(-2pt,2pt)$) and ($(texte.south east)+(-3pt,1pt)$) .. ($(texte.south east)+(-5pt,0)$) .. controls ($(texte.south east)+(-4.5pt,2.5pt)$) .. ($(texte.south east)+(-5pt,5pt)$) .. controls ($(texte.south east)+(-2.5pt,4.5pt)$) .. ($(texte.south east)+(0,5pt)$);
\node at(texte) {\usebox{\rem@box}};
\node[above right,fill=remtitle@bg@color,text=remtitle@color] at (texte.north west) {\remtitle@font \titletext\pluriel@rem \ifnum\thebreak@box>0 ~(suite)\fi};
\end{tikzpicture}
}

%%%%%%%%%%%%% Environnement "methode"

\newsavebox{\meth@box}
\newenvironment{methode}[1][]
{%
\setcounter{break@box}{0}
\setcounter{type@box}{0}
\typebox{meth}
\def\titletext{M\'ethode}
\ifx#1\@empty\xdef\pluriel@meth{}\else\xdef\pluriel@meth{#1}\fi
\itemclass{methtitle@bg@color}{\methtitle@font}
\begin{lrbox}{\meth@box}
\begin{minipage}{\linewidth}
}
{%
\end{minipage}
\end{lrbox}
\vspace*{\spacebeforemeth}
\begin{tikzpicture}
\node (texte) {\usebox{\meth@box}\vspace*{\spacebottommeth}};
\fill[fill=shadow@color] ($(texte.south west)+(1pt,-1pt)$) -- ($(texte.north west)+(1pt,-1pt)$) -- ($(texte.north east)+(1pt,-1pt)$) -- ($(texte.south east)+(0,6.5pt)+(1pt,-2pt)$) .. controls ($(texte.south east)+(-2pt,2pt)+(1pt,-1pt)$) and ($(texte.south east)+(-3pt,1pt)+(1pt,-1pt)$) .. ($(texte.south east)+(-6.5pt,0)+(1pt,-1pt)$) -- cycle;
\fill[fill=meth@bg@color] (texte.south west) -- (texte.north west) -- (texte.north east) -- ($(texte.south east)+(0,5pt)$) -- ($(texte.south east)+(-5pt,0)$) -- cycle;
\fill[fill=shadow@color!50!black] ($(texte.south east)+(0,5pt)$) .. controls ($(texte.south east)+(-2pt,2pt)$) and ($(texte.south east)+(-3pt,1pt)$) .. ($(texte.south east)+(-5pt,0)$) .. controls ($(texte.south east)+(-4.5pt,2.5pt)$) .. ($(texte.south east)+(-5pt,5pt)$) .. controls ($(texte.south east)+(-2.5pt,4.5pt)$) .. ($(texte.south east)+(0,5pt)$);
\node at(texte) {\usebox{\meth@box}};
\node[above right,fill=methtitle@bg@color,text=methtitle@color] at (texte.north west) {\methtitle@font \titletext\pluriel@meth \ifnum\thebreak@box>0 ~(suite)\fi};
\end{tikzpicture}
}

%%%%%%%%%%% Environnement "definition"

\newsavebox{\def@box}
\newenvironment{definition}[1][]
{%
	\setcounter{break@box}{0}
	\setcounter{type@box}{1}
	\typebox{def}
	\def\titletext{D\'efinition}
	\ifx#1\@empty\xdef\pluriel@def{}\else\xdef\pluriel@def{#1}\fi
	\itemclass{deftitle@bg@color}{\deftitle@font}
	\begin{lrbox}{\def@box}
	\begin{minipage}{\dimexpr\linewidth-\marginleft}
}
{%
	\end{minipage}
	\end{lrbox}
	\vspace*{\spacebeforedef}
	\begin{tikzpicture}
	\node[fill=def@bg@color] (texte) {\usebox{\def@box}};
	\node[fill=deftitle@bg@color,text=deftitle@color,below left,draw=deftitle@bg@color,very thick,text width={\dimexpr\marginleft-2mm},align=center] at ($(texte.north west)+(1pt,0)$) {\deftitle@font \titletext\pluriel@def \ifnum\thebreak@box>0\par(suite)\fi};
	\end{tikzpicture}
}

%%%%%%%%%%% Environnement "propriete"

\newsavebox{\prop@box}
\newenvironment{propriete}[1][]
{%
	\setcounter{break@box}{0}
	\setcounter{type@box}{1}
	\typebox{prop}
	\def\titletext{Propri\'et\'e}
	\ifx#1\@empty\xdef\pluriel@prop{}\else\xdef\pluriel@prop{#1}\fi
	\itemclass{proptitle@bg@color}{\proptitle@font}
	\begin{lrbox}{\prop@box}
	\begin{minipage}{\dimexpr\linewidth-\marginleft}
}
{%
	\end{minipage}
	\end{lrbox}
	\vspace*{\spacebeforeprop}
	\begin{tikzpicture}
	\node[fill=prop@bg@color] (texte) {\usebox{\prop@box}};
	\node[fill=proptitle@bg@color,text=proptitle@color,below left,draw=proptitle@bg@color,very thick,text width={\dimexpr\marginleft-2mm},align=center] at ($(texte.north west)+(1pt,0)$) {\proptitle@font \titletext\pluriel@prop \ifnum\thebreak@box>0\par(suite)\fi};
	\end{tikzpicture}
}

%%%%%%%%%%% Environnement "theoreme"

\newsavebox{\thm@box}
\newenvironment{theoreme}[1][]
{%
	\setcounter{break@box}{0}
	\setcounter{type@box}{1}
	\typebox{thm}
	\def\titletext{Th\'eor\`eme}
	\ifx#1\@empty\xdef\pluriel@thm{}\else\xdef\pluriel@thm{#1}\fi
	\itemclass{thmtitle@bg@color}{\thmtitle@font}
	\begin{lrbox}{\thm@box}
	\begin{minipage}{\dimexpr\linewidth-\marginleft}
}
{%
	\end{minipage}
	\end{lrbox}
	\vspace*{\spacebeforethm}
	\begin{tikzpicture}
	\node[fill=thm@bg@color] (texte) {\usebox{\thm@box}};
	\node[fill=thmtitle@bg@color,text=thmtitle@color,below left,draw=thmtitle@bg@color,very thick,text width={\dimexpr\marginleft-2mm},align=center] at ($(texte.north west)+(1pt,0)$) {\thmtitle@font \titletext\pluriel@prop \ifnum\thebreak@box>0\par(suite)\fi};
	\end{tikzpicture}
}

%%%%%%%%%%% Environnement "exemple"

\newsavebox{\ex@box}
\newenvironment{exemple}[1][]
{%
	\setcounter{break@box}{0}
	\setcounter{type@box}{1}
	\typebox{ex}
	\def\titletext{Exemple}
	\ifx#1\@empty\xdef\pluriel@ex{}\else\xdef\pluriel@ex{#1}\fi
	\itemclass{extitle@bg@color}{\extitle@font}
	\begin{lrbox}{\ex@box}
	\begin{minipage}{\dimexpr\linewidth-\marginleft}
}
{%
	\end{minipage}
	\end{lrbox}
	\vspace*{\spacebeforeex}
	\begin{tikzpicture}
	\node[fill=ex@bg@color] (texte) {\usebox{\ex@box}};
	\node[fill=extitle@bg@color,text=extitle@color,below left,draw=extitle@bg@color,very thick,text width={\dimexpr\marginleft-2mm},align=center] at ($(texte.north west)+(1pt,0)$) {\extitle@font \titletext\pluriel@ex \ifnum\thebreak@box>0\par(suite)\fi};
	\end{tikzpicture}
}

%%%%%%%%%%% Environnement "demonstration"

\newsavebox{\dem@box}
\newenvironment{demonstration}[1][1]
{%
	\setcounter{break@box}{0}
	\setcounter{type@box}{2}
	\typebox{dem}
	\def\titletext{D\'emonstration}
	\ifx#1\@empty\xdef\indic{0}\else\xdef\indic{#1}\fi
	\xdef\indic{#1}
	\begin{lrbox}{\dem@box}
	\begin{minipage}{\dimexpr\linewidth-0.5\marginleft}
	{\demtitle@font \titletext}\par\medskip%
}
{%
	\vskip-1em\hfill$\blacksquare$
	\end{minipage}
	\end{lrbox}
	\vspace*{\spacebeforedem}
	\begin{tikzpicture}
	\node[fill=dem@bg@color] (texte) {\usebox{\dem@box}};
	\ifnum\indic=1
		\draw[thick,draw=dem@border@color,decorate, decoration={random steps, segment length=3pt,amplitude=1pt}] (texte.north west) -- (texte.south west);
	\else
		\ifnum\indic=2
			\draw[thick,draw=dem@border@color,decorate, decoration={random steps, segment length=3pt,amplitude=1pt}] (texte.north west) -- (texte.south west) -- (texte.south east);
		\else
			\ifnum\indic=3
				\draw[thick,draw=dem@border@color,decorate, decoration={random steps, segment length=3pt,amplitude=1pt}] (texte.north west) -- (texte.south west);
				\draw[thick,draw=dem@border@color,decorate, decoration={random steps, segment length=3pt,amplitude=1pt}] (texte.north east) -- (texte.south east);
			\else
				\ifnum\indic=4
					\draw[thick,draw=dem@border@color,decorate, decoration={random steps, segment length=3pt,amplitude=1pt}] (texte.north east) -- (texte.north west) -- (texte.south west);
				\else
					\ifnum\indic=5
						\draw[thick,draw=dem@border@color,decorate, decoration={random steps, segment length=3pt,amplitude=1pt}] (texte.north west) -- (texte.south west) -- (texte.south east) -- (texte.north east) -- cycle;
					\fi
				\fi
			\fi
		\fi
	\fi
	\node[below left,minimum width=0.5\marginleft] at (texte.north west) {\phantom{-}};
	\end{tikzpicture}
}

%%%%%%%%%%%% Activité

\newcounter{activity}[chapter]

\newcommand{\activite}[1]
{%
	\refstepcounter{activity}
	\vspace*{\spacebeforeact}
	\begin{tikzpicture}
	\fill[shadow@color] (0.1,-.15) circle[x radius=1.5em,y radius=0.66em,rotate=10];
	\fill[activitytitle@color,rotate=30] (0,0) circle[x radius=1.5em,y radius=1em];
	\node[text=activitynum@color] at (0,0) {\activity@font\theactivity};
	\node[right,text=activitytitle@color] at (0.75,0) {\activity@font#1};
	\end{tikzpicture}
	\itemclass{activitytitle@color}{\activity@font}
}

%%%%%%%%%%%%%%%%%% Style des itemize

\newcommand{\itemclass}[2] % #1 = couleur ; #2 = fonte
{
	\setlist[itemize,1]{label={\color{#1}\textbullet}}
	\setlist[itemize,2]{label={\color{#1}$\rightarrow$}}
	\setlist[enumerate,1]
	{%
		label=\fcolorbox{#1}
		{#1!20}
		{\color{#1}#2\normalsize\arabic*}
	}
	\setlist[enumerate,2]{label=\textcolor{#1}{#2\normalsize\alph*.}}
}
\itemclass{black}{}

%%%%%%%%%%%%%%%%%%%%%%%%%%%%%%%%%%%%%%%%%%%%%%%%%%%%%%%%%%% Exercices

% Déclaration de la 1ère page d'exercices

\newcounter{indicCorr}
\newcommand{\exostart}[1][0]
{%
	\newpage
	\pagecolor{exercices@bg@color}
	\begin{flushright}
	\begin{tikzpicture}
	\node[inner sep=1em] (titre) {\exercicetitle@font Exercices};
	\fill[rounded corners=2pt,fill=shadow@color] ($(titre.north west)+(2pt,-2pt)$) -- ($(titre.north east)+(1mm,0mm)+(2pt,-2pt)$) -- ($(titre.south east)+(2pt,-2pt)$) -- ($(titre.south west)+(-1mm,0)+(2pt,-2pt)$) -- cycle;
	\fill[rounded corners=2pt,fill=exercicetitle@bg@color] (titre.north west) -- ($(titre.north east)+(1mm,0mm)$) -- (titre.south east) -- ($(titre.south west)+(-1mm,0)$) -- cycle;
	\node[inner sep=1em,text=exercicetitle@color] at (titre) {\exercicetitle@font Exercices};
	\end{tikzpicture}
	\end{flushright}
	\ifnum#1=1
		\setcounter{indicCorr}{1}
	\else
		\setcounter{indicCorr}{0}
	\fi
}

% Environnement "exercice"

\newcounter{exercice}[chapter]
\newcounter{chapterant}

\newsavebox{\exo@box}
\newenvironment{exercice}[1][0]
{%
	\ifnum#1=0
		\refstepcounter{exercice}
		\expandafter\xdef\csname nbexos\thechapter\endcsname{\theexercice}
		\ifnum\theexercice<10
			\def\num@exo{0\theexercice}
		\else
			\def\num@exo{\theexercice}
		\fi
	\else
		\xdef\num@exo{}
	\fi
	\itemclass{exercicenum@bg@color}{\exercicenum@font}
	\begin{lrbox}{\exo@box}
	\begin{minipage}{\dimexpr\linewidth-3em-10pt}
}
{%
	\ifnum\theindicCorr=1
	\begin{flushright}
	\textcolor{corref@color}{\corref@font \scriptsize Corrigé p \normalsize \pageref{corrige-\thechapter-\theexercice}}
	\end{flushright}	
	\fi
	\end{minipage}
	\end{lrbox}
	\ifnum\theexercice>1
		\ifx\num@exo\@empty
		\else
			\par\vspace*{\spacebeforeexo}
		\fi
	\fi
	\begin{tikzpicture}
	\node (texte) {\usebox{\exo@box}};
	\ifx\num@exo\@empty
		\node[below left,minimum width=2em] at ($(texte.north west)+(-0.5em,0)$) {\phantom{\exercicenum@font\num@exo}};
	\else
		\node[below left,minimum width=2em] (num) at ($(texte.north west)+(-0.5em,0)$) {\exercicenum@font\num@exo};
		\fill[rounded corners=2pt,fill=exercicenum@bg@color] (num.north west) -- ($(num.north east)+(1mm,0mm)$) -- (num.south east) -- ($(num.south west)+(-1mm,0)$) -- cycle;
		\node[text=exercicenum@color] at (num) {\exercicenum@font\num@exo};
	\fi
	\end{tikzpicture}	
}

%%%%%%%%%%%%%%% Corrigés
% Déclaration de la 1ère page de corrections

\newcommand{\corrstart}
{%
	\newpage
	\pagecolor{corriges@bg@color}
	\begin{flushright}
	\begin{tikzpicture}
	\node[inner sep=1em] (titre) {\corrigetitle@font Corrig\'es des exercices};
	\fill[rounded corners=2pt,fill=shadow@color] ($(titre.north west)+(2pt,-2pt)$) -- ($(titre.north east)+(1mm,0mm)+(2pt,-2pt)$) -- ($(titre.south east)+(2pt,-2pt)$) -- ($(titre.south west)+(-1mm,0)+(2pt,-2pt)$) -- cycle;
	\fill[rounded corners=2pt,fill=corrigetitle@bg@color] (titre.north west) -- ($(titre.north east)+(1mm,0mm)$) -- (titre.south east) -- ($(titre.south west)+(-1mm,0)$) -- cycle;
	\node[inner sep=1em,text=corrigetitle@color] at (titre) {\corrigetitle@font Corrig\'es des exercices};
	\end{tikzpicture}
	\end{flushright}
}

% BreakCorr

\newcommand{\BreakCorr}[1][\newpage]
{%
\end{corrige}
#1
\begin{corrige}[\thechapter]{\i}
\bgroup\exercicenum@font\textbf{(suite)}\egroup~
}

\newcommand{\AfficheCorriges}[1][]
	{%
		\multido{\i=1+1}{\theexercice}{%
			\foreach\x/\macr in {#1}%
			{%
				\ifnum\x=\i\macr\fi%
			}%
			\input{tmp/corriges/\thechapter-\i}%
		}
	}

%% Portion de code pour écrire dans un fichier %% Merci pg ...

\newwrite{\output@stream@corrige}

\newcommand{\OuvrirFlux}[1]
{% ouvre le flux de fichier
  \immediate\openout\output@stream@corrige #1%
}
\newcommand{\FermerFlux}
{% ferme le flux de fichier
  \immediate\closeout\output@stream@corrige%
}
\newcommand{\EcritureFlux}[1]
{% écrit un court texte dans le fichier
  \immediate\write\output@stream@corrige{\unexpanded{#1}}%
}
\newcommand{\CommencerExportFlux}
{% commence l'écriture prolongée
   \bgroup\@bsphack
   \let\do\@makeother\dospecials
   \catcode`\^^M\active
   \def\verbatim@processline{%
     \immediate\write\output@stream@corrige{\the\verbatim@line}}%
   \verbatim@start}
\ifdefined\ArreterExportFlux % arrête l'écriture prolongée
  \newcommand{\ArreterExportFlux}{\@esphack\egroup}
\else
  \def\ArreterExportFlux{\@esphack\egroup}
\fi

\newenvironment{correction}
  {\OuvrirFlux{tmp/corriges/\thechapter-\theexercice.tex}%
   \EcritureFlux{\begin{corrige}[\thechapter]{\i}}%
   \CommencerExportFlux}
  {\ArreterExportFlux%
   \EcritureFlux{\end{corrige}\par\vspace*{\spacebeforeexo}}%
   \FermerFlux}

\newsavebox{\corr@box}
\newenvironment{corrige}[2][]
{%
	\label{corrige-#1-#2}%
	\ifnum#2<10%
		\def\num@exo{0#2}%
	\else%
		\def\num@exo{#2}%
	\fi%
	\itemclass{corrigenum@bg@color}{\exercicenum@font}%
	\begin{lrbox}{\corr@box}
	\begin{minipage}{\dimexpr\linewidth-3em-10pt}
}
{%
	\end{minipage}
	\end{lrbox}
	\begin{tikzpicture}
	\node (texte) {\usebox{\corr@box}};
	\ifx\num@exo\@empty
		\node[below left,minimum width=2em] at ($(texte.north west)+(-0.5em,0)$) {\phantom{\exercicenum@font\num@exo}};
	\else
		\node[below left,minimum width=2em] (num) at ($(texte.north west)+(-0.5em,0)$) {\exercicenum@font\num@exo};
		\fill[rounded corners=2pt,fill=corrigenum@bg@color] (num.north west) -- ($(num.north east)+(1mm,0mm)$) -- (num.south east) -- ($(num.south west)+(-1mm,0)$) -- cycle;
		\node[text=corrigenum@color] at (num) {\exercicenum@font\num@exo};
	\fi
	\end{tikzpicture}
}

%%%%%%%%%%%%%%% INDEX

\newcommand\printindex{%
	\phantomsection\addcontentsline{toc}{chapter}{\indexname}
	\@input@{\jobname.ind}
}
\def\nbcolindex{2}
\renewenvironment{theindex}
{%
	\if@twocolumn
		\@restonecolfalse
	\else
		\@restonecoltrue
	\fi
	\newpage
	\begin{flushright}\begin{tikzpicture}
	\node[fill=indextitle@bg@color,text=indextitle@color, xslant=0.2,rounded corners=2pt,inner xsep=1em,inner ysep=1ex] {\indextitle@font\indexname};
	\end{tikzpicture}\end{flushright}
	\pagecolor{white}
	\thispagestyle{main}\parindent\z@
	\let\item\@idxitem
	\columnseprule \z@
	\columnsep 35\p@
	\ifnum\nbcolindex>1
	\begin{multicols}{\nbcolindex}
	\fi
}
{%
	\ifnum\nbcolindex>1
	\end{multicols}
	\fi
	\if@restonecol\onecolumn\else\clearpage\fi
}

%%%%%%%%%%%%% TOC

\renewcommand{\contentsname}{Sommaires}
\renewcommand\tableofcontents
{%
	\thispagestyle{toc}
	\AddToShipoutPicture
  	{%
		  %\put(\LenToUnit{0mm},\LenToUnit{0mm})
		\put(\LenToUnit{0.05\paperwidth},\LenToUnit{.96\paperheight})
  		{%
			\begin{tikzpicture}
				%\clip (0,0) rectangle+(\paperwidth,-\paperheight);
	    		%\draw[decorate,decoration={text along path,text={|\contentstitle@font\color{contentstitleshadow@color}|\contentsname}}] (1.6,-8.1) .. controls ($(\controltoctitle,-\controltoctitle)+(0.1,-0.1)$) .. (20.1,-2.1);
				%\draw[decorate,decoration={text along path,text={|\contentstitle@font\color{contentstitle@color}|\contentsname}}] (1.5,-8) .. controls (\controltoctitle,-\controltoctitle) .. (20,-2);
				\draw[ultra thick] node[font=\fontsize{20}{20}\selectfont\bfseries] {\contentsname};% moved here, changed font shape				
			\end{tikzpicture}
		}
	}
	\if@twocolumn
          \@restonecoltrue\onecolumn
        \else
          \@restonecolfalse
        \fi
        \@starttoc{toc}%
        \if@restonecol\twocolumn\fi
        }

